%%% LaTeX Template: Curriculum Vitae
%%%
%%% Source: http://www.howtotex.com/
%%% Feel free to distribute this template, but please keep the referal to HowToTeX.com
%%% Date: July 2011

%%% ------------------------------------------------------------
%%% BEGIN PREAMBLE
%%% ------------------------------------------------------------
\documentclass[paper=a4,fontsize=11pt]{scrartcl}	

%\usepackage{fontspec}
%\usepackage{xunicode}
%\usepackage{xltxtra}
%\usepackage[boldfont,slantfont,CJKsetspaces,CJKchecksingle]{xeCJK}

							
%\usepackage[english]{babel}
%\usepackage[protrusion=true,expansion=true]{microtype}
\usepackage{amsmath,amsfonts,amsthm}
\usepackage[pdftex]{graphicx}
\usepackage[svgnames]{xcolor}
\usepackage[top=15ex, bottom=15ex, left=15ex, right=15ex]{geometry}
	\textheight=700px
\usepackage{url}
\usepackage{wrapfig}
\usepackage{hyperref}
\usepackage{url}

\frenchspacing
\pagestyle{empty}		% No pagenumbers/headers/footers
%\usepackage{bbding}	% Symbols


\usepackage{sectsty}	% Custom sectioning

\sectionfont{%			% Change font of \section command
	\usefont{OT1}{phv}{b}{n}%			% bch-b-n: CharterBT-Bold font
	\sectionrule{0pt}{0pt}{-5pt}{1.5pt}
	}

\newlength{\spacebox}
\settowidth{\spacebox}{8888888888}			% Box to align text
\newcommand{\sepspace}{\vspace*{1em}}		% Vertical space macro

\newcommand{\MyName}[1]{
		\Huge \usefont{OT1}{phv}{b}{n} \hfill #1 		% Name
		\par \normalsize \normalfont}
		
\newcommand{\MySlogan}[1]{
		\large \usefont{OT1}{phv}{m}{n}\hfill \textit{#1} % Slogan (optional)
		\par \normalsize \normalfont}

\newcommand{\NewPart}[1]{\section*{\uppercase{#1}}}

\newcommand{\TableEntry}[2]{
		\noindent\hangindent=2em\hangafter=0 	% Indentation
		\parbox{\spacebox}{
		\textit{#1}}							% Title (birth, address, skills, etc.)
		\hspace{1.5em} #2 \par}					% Entry

\newcommand{\DetailEntry}[4]{
		\noindent \hangindent=2em\hangafter=0 \textbf{#1} \hfill 	% Title
		\colorbox{Grey}{%
			\parbox{8.1em}{%
			\hfill\color{White}#2}} \par						% Time
		\vspace{0.2em}
		\noindent \hangindent=3em\hangafter=0\textit{#3} \par		% Place, School, Company, etc.
		\noindent\hangindent=3em\hangafter=0 \small #4 			% Description
		\normalsize \par}

\newcommand{\BasicEntry}[3]{
		\noindent \hangindent=2em\hangafter=0 \textbf{#1} \hfill 	% Title
		\colorbox{Grey}{%
			\parbox{8.1em}{%
			\hfill\color{White}#2}} \par						% Time
		\vspace{0.2em}
		\noindent\hangindent=3em\hangafter=0 \small #3 			% Description
		\normalsize \par}

\newcommand{\SimpleEntry}[2]{
		\noindent \hangindent=2em\hangafter=0 {#1} \hfill 	% Description
		\colorbox{Grey}{%
			\parbox{3.6em}{%
			\hfill\color{White}#2}} \par					% Time
		\normalsize \par}

\newcommand{\WordsEntry}[1]{
		\noindent \hangindent=2em\hangafter=0 {#1} \hfill 	% Words
		\normalsize \par}

%%% ------------------------------------------------------------
%%% BEGIN DOCUMENT
%%% ------------------------------------------------------------

\begin{document}
%\begin{wrapfigure}{l}{0.5\textwidth}						%Insert a picture if needed
%	\vspace*{-2em}
%		\includegraphics[width=0.15\textwidth]{picture}
%\end{wrapfigure}

\MyName{张柘}
\MySlogan{个人简历}

\sepspace



\section*{Personal Data}

\begin{tabular}{rl}
\textsc{性别:} & 男 \\
\textsc{生日:} & 1988.01.31 \\
%\textsc{Place of birth:} & Luoyang, China \\
\textsc{地址:} &  Room 3512, Nguyen Engineering Building, George Mason University, \\
	& 4400 University Drive, Fairfax, VA 22030-4444. \\
\textsc{电话:} & +1-(202)531-7210\\
\textsc{Email:} & \href{mailto:zzhang18@gmu.edu}{zzhang18@gmu.edu}; \href{mailto:nagatokana@gmail.com}{nagatokana@gmail.com} 
\end{tabular}

%----------------------------------------------------------------------------------------
%	WORKING
%----------------------------------------------------------------------------------------

\section*{工作经历}

\begin{tabular}{r|p{11cm}}

\emph{2019.8--} & Post-Doctoral Affiliate(博士后合作研究员)\\	
\emph{2016.12--2019.8} & Post-Doctoral Research Fellow (博士后研究员)\\
& \normalsize\textbf{乔治梅森大学}, Fairfax, VA 22030, USA\\
%  & Project: ``Error Analysis in Sparse Signal Processing and Synthetic Aperture Sonar'' \\
	\multicolumn{2}{c}{} \\


  \emph{2015.12--2016.11} & Post-Doctoral Research Scientist (博士后研究员)\\
  & \normalsize\textbf{乔治华盛顿大学}, Washington, DC 20052, USA\\
%  & Project: ``Error Analysis in Sparse Signal Processing and Synthetic Aperture Sonar'' \\
\multicolumn{2}{c}{} \\

  \emph{2006.4--2015.12} & 最高技术负责人\\
& \normalsize\textbf{西安交通大学兵马俑BBS}\\
%  & Project: ``Error Analysis in Sparse Signal Processing and Synthetic Aperture Sonar'' \\
\multicolumn{2}{c}{} \\

  \emph{2010.06--2019.02} & 最高技术负责人\\
& \normalsize\textbf{中国科学院科苑星空BBS}\\
%  & Project: ``Error Analysis in Sparse Signal Processing and Synthetic Aperture Sonar'' \\
	

\end{tabular}

%----------------------------------------------------------------------------------------
%	EDUCATION
%----------------------------------------------------------------------------------------
\section*{教育经历}

\begin{tabular}{r|p{11cm}}	
	\emph{2009.09--2015.07} & 工学博士 \textsc{信号与信息处理} \\
	& \normalsize\textbf{中国科学院电子学研究所}, 北京, 中国\\
	& \normalsize\textbf{中国科学院大学}, 北京, 中国\\
	%  & \normalsize \textsc{Gpa}: 3.7/4.0\\ 
	%  & \\
	\multicolumn{2}{c}{} \\
	
	%------------------------------------------------
	
	\emph{2014.01--2014.04} & 访问学生 \\
	& \normalsize\textbf{康涅狄格大学}, Storrs, CT, USA\\
	\multicolumn{2}{c}{} \\


	\emph{2004.09--2008.07} & 工学学士 \textsc{}\textsc{信息工程} \\
	& \small\emph{电子与信息工程学院}\\
	& \normalsize\textbf{西安交通大学}, 西安, 中国\\
	%  & \normalsize \textsc{Gpa}: 3.4/4.0\\
	\multicolumn{2}{c}{} \\
	
	%%------------------------------------------------
	
	\emph{2003.09--2004.07} & \textsc{}\textsc{少年班} \\
	  & \normalsize\textbf{西安交通大学}, 西安, 中国

%------------------------------------------------



\end{tabular}

\section*{主持开源项目}

\begin{tabular}{r|p{11cm}}
	\emph{2006.04--2015.11} & 西安交通大学兵马俑BBS (\href{http://bbs.xjtu.edu.cn/}{http://bbs.xjtu.edu.cn/}, \href{http://bmybbs.com/}{http://bmybbs.com/}) \\
	& 项目托管: \href{https://github.com/bmybbs?type=source}{https://github.com/bmybbs?type=source} \\
	& \emph{最高技术负责人} \\ 
	& \footnotesize{兵马俑BBS是中国教育网最大的BBS站点之一,属于西安交通大学. 兵马俑BBS服务50000多名用户,同时在线数最高5000. 项目代码超过10万行.}\\
	\multicolumn{2}{c}{} \\
	
	%------------------------------------------------
	
	\emph{2010.06--2019.04} & 中国科学院科苑星空BBS (\href{http://bbs.ucas.ac.cn/}{http://bbs.ucas.ac.cn/}, \href{http://kyxk.net/}{http://kyxk.net/}) \\
	& \emph{最高技术负责人} \\ 
	& \footnotesize{科苑星空BBS是中国科学院的官方社区. 科苑星空BBS服务30000多名用户,同时在线数最高1500. 项目代码超过12万行.}\\
	\multicolumn{2}{c}{} \\
	
	\emph{2018.9--} & hCNN项目 \\
	& 项目托管: \href{https://github.com/pzhg/hCNN}{https://github.com/pzhg/hCNN} \\
	& \emph{唯一作者} \\ 
	& \footnotesize{hCNN是一个开源的深度学习框架,可方便的进行深度学习开发与训练,并与传统信号处理方法兼容.}\\
	\multicolumn{2}{c}{} \\
\end{tabular}

%----------------------------------------------------------------------------------------
%	RESEARCH EXPERIENCE
%----------------------------------------------------------------------------------------




%----------------------------------------------------------------------------------------
%	COMPUTER SKILLS 
%----------------------------------------------------------------------------------------

\section*{专业技能}

\begin{tabular}{rl}

专业方向: & 深度学习/AI.\\
	& 机器学习与信号处理方法的结合.\\
	& 稀疏信号处理与低秩优化.\\
	& 合成孔径雷达.\\

专业技能: & 大型在线社区开发、运营与维护,Linux运维,GNU工具链.\\
 	& 专业编程能力:C/C++/Python/JavaScript.\\
 	& 专业软件:MATLAB.
\end{tabular}
%----------------------------------------------------------------------------------------
%	SCHOLARSHIPS AND ADDITIONAL INFO
%----------------------------------------------------------------------------------------

\section*{研究经历}

\begin{tabular}{r|p{11cm}}
	
	\emph{2015.12--} & \textbf{Task-Cognizant Sparse Sensing for Inference} \\
	& 资助: \emph{National Science Foundation (NSF)}\\
	% & Key research member \\
	\multicolumn{2}{c}{} \\
	
	% : CIF: Small: Task-Cognizant Sparse Sensing for Inference
	%------------------------------------------------
	
	
	\emph{2010.04--2015.07} & \textbf{稀疏微波成像的理论、系统与方法研究} \\
	& 资助: \emph{国家重大基础科学研究计划(973计划)}\\
	% & Key research member \\
	\multicolumn{2}{c}{} \\
	
	\emph{2012.04--2015.07} & \textbf{现金微波探测予信息处理} \\
	& 资助: \emph{中国科学院} and \emph{国家外国专家局}\\
	\multicolumn{2}{c}{} \\
	
	%------------------------------------------------
	
	\emph{2007.09--2008.07} & \textbf{矢量量化技术的研究与应用} \\
	& 资助: \emph{国家自然科学基金 (NSFC)}\\
	% & Research member \\
	
	%------------------------------------------------
	
\end{tabular}

\section*{专业服务}

\textbf{期刊审稿人} 
\begin{itemize}
	\item \textit{IEEE Transactions on Signal Processing} Journal.
	\item \textit{IET Radar, Sonar \& Navigation} Journal.
	\item \textit{Electronics Letters} Journal.
	\item \textit{IET Signal Processing} Journal.
\end{itemize}
~\\
\textbf{学术会议学术委员会 (TPC) 成员 / 审稿人} 
\begin{itemize}
	\item 5th Int. Workshop on Compressed Sensing Theory and its Applications to Radar, Sonar and Remote Sensing (CoSeRa 2018).
	\item 4th International Conference on Electrical Engineering, Computer Science and Informatics (EECSI 2017, Grand Mercure, Yogyakarta, Indonesia).
	\item 1st Annual Saudi Interventional Radiology Society Conference (SIRS 2017, Jeddah, Kingdom of Saudi Arabia).
	\item 4th Int. Workshop on Compressed Sensing Theory and its Applications to Radar, Sonar and Remote Sensing (CoSeRa 2016, Aachen, Germany).
	\item 3rd Int. Workshop on Compressed Sensing Theory and its Applications to Radar, Sonar and Remote Sensing (CoSeRa 2015, Pisa, Italy).
	\item 1st International Conference on Single Processing and Data Mining (ICSPDM 2015, Istanbul, Turkey).
	\item 15th IEEE International Symposium on Signal Processing and Information Technology (ISSPIT 2015, Abu Dhabi, UAE).
	\item 以及 DISP 2019, FSDM 2018, SIRS 2018, EECSI 2018, ICW-TELKOMNIKA 2018, ICITech 2017, FSDM 2017, ADICS-ESIT 2020, BEEI 2020, SIRS'19, IJCDS-2020, TELKOMNIKA, SIRS'20, FSDM2020 等学术会议. 
\end{itemize}

\section*{获奖}

\begin{tabular}{r|p{11cm}}

\emph{2008.04} & 德州仪器(TI)DSP大奖赛 \\
& \emph{优胜奖}\\
\multicolumn{2}{c}{} \\

%------------------------------------------------

\emph{2008.02} & 美国大学生数学建模竞赛\\
& \emph{一等奖}\\

%------------------------------------------------

\emph{2007.02} & 美国大学生数学建模竞赛\\
& \emph{二等奖}\\
\multicolumn{2}{c}{} \\

%------------------------------------------------

\emph{2006.09} & 中国大学生数学建模竞赛\\
& \emph{二等奖}\\
\multicolumn{2}{c}{} \\

%------------------------------------------------

\end{tabular}

%----------------------------------------------------------------------------------------
%	SCHOLARSHIPS AND ADDITIONAL INFO
%----------------------------------------------------------------------------------------

%\section*{Scholarships}
%
%\begin{tabular}{rl}
%2004 & Freshman Scholarship with First Level\\
%
%2007 & Siyuan Scholarship of Xi'an Jiaotong University\\
%
%2013 & Merit Student Award of University of Chinese Academy of Sciences\\ 
%\end{tabular}



%----------------------------------------------------------------------------------------
%	WORK EXPERIENCE 
%----------------------------------------------------------------------------------------



%----------------------------------------------------------------------------------------
%	LANGUAGES
%----------------------------------------------------------------------------------------

\section*{语言}

\begin{tabular}{rl}

\textsc{英语:} & 流利,专业,在美生活5年 (CET-6 pass, PETS-5 pass, WSK pass, TOEFL 96) \\

\textsc{日语:} & 可交流\\
\end{tabular}

%----------------------------------------------------------------------------------------
%	INTERESTS AND ACTIVITIES
%----------------------------------------------------------------------------------------

\section*{兴趣}

IT, 开源, 编程, Linux\\
音乐, 动漫, 历史, 常旅客

%----------------------------------------------------------------------------------------

\section*{发表}

~\\

\textsc{Book chapter}

\begin{description}
\item B. Zhang, W. Hong, Y. Wu, \underline{Z. Zhang} and C. Jiang, ``System Design and Signal Processing of Synthetic Aperture Radar with Sparse Constraint,'' in \textit{Sparse Reconstruction and Compressive Sensing in Remote Sensing}, Springer, to publish.
\end{description}

~\\

\textsc{Journal Articles}

(Bold for corresponding / communication author)

\begin{description}
	
% \item \textbf{\underline{Z. Zhang$\ast$} and Z. Tian, ``Efficient Two-Dimensional Harmonic Retrieval Based on Decoupled Atomic Norm Minimization,'' IEEE Signal Processing Letter, submitted}.



\item \textbf{\underline{Z. Zhang$\ast$} and Z. Tian, ``Car-borne Synthetic Aperture Radar (SAR) Automatic Target Identification (ATR) for Autonomous Driving based on a Novel Multi-level Neural Network Architecture,'' to submit}.
	
% \item \textbf{\underline{Z. Zhang$\ast$} and Z. Tian, ``Multi-level multi-channel Neural Network: An Architecture for Deep Learning from Multiple Aspects of Data and its Application in Signal Processing,'' to submit}.

%\item \textbf{\underline{Z. Zhang$\ast$} and Z. Tian, ``ANM-PhaseLift: Efficient Structured Line Spectrum Estimation from Quadratic Measurements,'' IEEE Transactions on Signal Processing, submitted}.

\item \textbf{\underline{Z. Zhang$\ast$}, Y. Wang, and Z. Tian, ``Efficient Two-Dimensional Line Spectrum Estimation Based on Decoupled Atomic Norm Minimization,'' Signal Processing, accepted}.

% \item \textbf{\underline{Z. Zhang$\ast$}, Y. Zhao, C. Jiang, B. Zhang, W. Hong and Y. Wu, ``An Autofocus Algorithm of Sparse Microwave Imaging Based on Phase Recovery Theory,'' Journal of Electronics and Information Technology, accepted, in Chinese}.

\item \textbf{\underline{Z. Zhang$\ast$}, B. Zhang, W. Hong and Y. Wu, ``Accelerated Error Compensation Algorithm of Sparse Microwave Imaging with Combination of Map-drift and SAR Raw Data Simulator,'' Journal of Radars, vol. 5, no. 1, pp. 25-34, 2016}.

\item \textbf{B. Zhang, \underline{Z. Zhang$\ast$}, C. Jiang, Y. Zhao, W. Hong and Y. Wu, ``System Design and First Airborne Experiment of Sparse Microwave Imaging Radar: Initial Results,'' Science China Information Sciences (Series F), vol. 58, no. 6, 2015.}

\item C. Jiang$\ast$, Y. Zhao, \underline{Z. Zhang}, B. Zhang, and W. Hong, ``Azimuth Sampling Optimization Scheme for Sparse Microwave Imaging Based on Mutual Coherence Criterion,'' Journal of Electronics and Information Technology, vol. 37, no. 3, 2015.

\item Y. Wu, W. Hong, B. Zhang$\ast$, C. Jiang, \underline{Z. Zhang} and Y. Zhao, ``Current Developments of Sparse Microwave Imaging,'' Journal of Radars, vol.3, no. 4, pp. 383--395, 2014, in Chinese.

\item C. Jiang$\ast$, B. Zhang, J. Fang, \underline{Z. Zhang}, W. Hong, Y. Wu and Z. Xu, ``An efficient Lq regularization algorithm with range-azimuth decoupled for SAR imaging,'' Electronics Letters, vol. 50, no. 3, pp. 204--205, 2014.

\item \textbf{\underline{Z. Zhang$\ast$}, B. Zhang, C. Jiang, Y. Xiang, W. Hong, and Y. Wu, ``Influence factors of sparse microwave imaging radar system performance: approaches to waveform design and platform motion analysis,'' Science China Information Sciences (Series F), vol. 55, no. 10, pp. 2301--2317, 2012}.

\item C. Jiang$\ast$, B. Zhang, \underline{Z. Zhang}, W. Hong, and Y. Wu, ``Experimental results and analysis of sparse microwave imaging from spaceborne radar raw data,'' Science China Information Sciences (Series F), vol. 55, no. 8, pp. 1801--1815, 2012.

% \item \textbf{\underline{Z. Zhang$\ast$}, B. Zhang, C. Jiang, X. Liang, L. Chen, W. Hong and Y. Wu, ``The First Airborne Experiment of Sparse Microwave Imaging: Prototype System Design and Result Analysis,'' to submit.}

\item M. Xie$\ast$, R. Qiao, Z. Pan, D. Li, Y. Qiao and \underline{Z. Zhang}, ``Realization of an Improved Absolute Error Inequality Algorithm on DM642,'' Microelectronics \& Computer, vol. 27, no. 4, pp. 182-185, 2010, in Chinese.

\item T. Wang$\ast$, R. Qiao, Z. Pan, D. Li, Y. Qiao, F. Gao and \underline{Z. Zhang}, ``Research and Application of Vector Quantization Algorithm Based on DM642'', in Proceedings of 2008 TI DSP Contest, pp. 143--161, Publishing House of Electronics Industry, Beijing, 2008, in Chinese. 

~\\

\end{description}

\textsc{Invited Peer-reviewed Conference Papers}

(Bold for corresponding / communication author)

\begin{description}
   
\item \textbf{\underline{Z. Zhang$\ast$}, B. Zhang, W. Hong, H. Bi and Y. Wu, ``SAR Imaging of Moving Target in a Sparse Scene Based on Sparse Constraints: Preliminary Experiment Results,'' in 2015 IEEE International Geoscience and Remote Sensing Symposium (IGARSS 2015), \emph{invited}.}

\item \textbf{W. Hong, B. Zhang, \underline{Z. Zhang$\ast$}, C. Jiang, Y. Zhao and Y. Wu, ``Radar Imaging with Sparse Constraint: Principle and Initial Experiment,'' in 10th European Conference on Synthetic Aperture Radar (EuSAR 2014), \emph{invited}}.

\end{description}

~\\

\textsc{Peer-reviewed Conference Papers}

(Bold for corresponding / communication author)

\begin{description}
	
\item {Z. Wang, X. Lin, X. Xiang, \underline{Z. Zhang}, Z. Tian, K. Pham, E. Blasch and G. Chen, ``A hidden chamber detector based on a MIMO SAR'', in Proc. SPIE 11017, Sensors and Systems for Space Applications XII, 1101706, 2019.}	
	
\item {P.Xu, Z. Tian, \underline{Z. Zhang$\ast$} and Y. Wang$\ast$, ``COKE: Communication-Censored Kernel Learning via random features'', in the 2019 IEEE Data Science Workshop (DSW 2019), 2019.}	
	
\item \textbf{\underline{Z. Zhang$\ast$}, X. Chen and Z. Tian$\ast$, ``A Hybrid Neural Network Framework and Application to Radar Automatic Target Recognition'', in the 6th IEEE Global Conference on Signal and Information Processing (GlobalSIP 2018), 2018.}

\item \textbf{\underline{Z. Zhang$\ast$} and Z. Tian$\ast$, ``ANM-PhaseLift: Structured Line Spectrum Estimation from Quadratic Measurements'', in 7th IEEE International Workshop on Computational Advances in Multi-Sensor Adaptive Processing (CAMSAP 2017), 2017.}

\item \textbf{Z. Tian$\ast$, \underline{Z. Zhang$\ast$} and Y. Wang, ``Low-complexity optimization for Two-Dimensional Direction-of-arrival Estimation via Decoupled Atomic Norm Minimizationg'', in 42th International Conference on Acoustics, Speech, and Signal Processing (ICASSP 2017), 2017.}

\item \textbf{\underline{Z. Zhang$\ast$}, Z. Tian, B. Zhang, W. Hong, W. Hong and L. Li, ``Multi-channel SAR Covariance Matrix Estimation Based on Compressive Covariance Sensing'', in 4th International Workshop on Compressive Sensng Theory and its Applications to Radar, Sonar and Remote Sensing (CoSeRa 2016), 2016.}

\item C. Jiang$\ast$, Y. Lin, \underline{Z. Zhang}, B. Zhang and W. Hong, ``WASAR Imaging based on message passing with structured sparse constraint: approach and experiment'', in 3th International Workshop on Compressive Sensng Theory and its Applications to Radar, Sonar and Remote Sensing (CoSeRa 2015), 2015.

\item X. Quan$\ast$, C. Jiang, \underline{Z. Zhang}, B. Zhang and Y. Wu, ``A Study of BP-CAMP Algorithm for SAR Imaging,'' in 2015 IEEE International Geoscience and Remote Sensing Symposium (IGARSS 2015), 2015.

\item X. Quan$\ast$, \underline{Z. Zhang}, C. Jiang, B. Zhang and Y. Wu, ``Comparison of Several Sparse Reconstruction Algorithms in SAR Imaging,'', in IET International Radar Conference 2015, 2015.

\item W. Wang$\ast$, B. Zhang, W. Hong, \underline{Z. Zhang}, Y. Zhao, C. Jiang and H. Bi, `` Polarimetric SAR Tomography of Forested Areas Based on Compressive MUSIC,'' in 2014 IEEE International Geoscience and Remote Sensing Symposium (IGARSS 2014), 2014.
	
\item \textbf{\underline{Z. Zhang$\ast$}, Y. Zhao, C. Jiang, B. Zhang, W. Hong and Y. Wu, ``Initial Analysis of SNR / Sampling Rate Constraints in Compressive Sensing based Imaging Radar,'' in 2nd Workshop on Compressive Sensng Applied to Radar (CoSeRa 2013), 2013}.

\item B. Zhang, C. Jiang$\ast$, \underline{Z. Zhang}, J. Fang, Y. Zhao, W. Hong, Y. Wu and Z. Xu, ``Azimuth Ambiguity Suppression for SAR Imaging based on Group Sparse Reconstruction'', in Workshop on Compressive Sensng Applied to Radar (CoSeRa 2013), 2013.
	
\item \textbf{\underline{Z. Zhang$\ast$}, Y. Zhao, C. Jiang, B. Zhang, W. Hong and Y. Wu, ``Autofocus of Sparse Microwave Imaging Radar Based on Phase Recovery,'' in 2nd IEEE International Conference on Signal Processing, Communications and Computing (ICSPCC 2013), 2013}.

\item \textbf{\underline{Z. Zhang$\ast$}, B. Zhang, W. Hong, and Y. Wu, ``Waveform Design for Lq Regularization Based Radar Imaging and An Approach to Radar Imaging with Non-moving Platform,'' in 9th European Conference on Synthetic Aperture Radar (EuSAR 2012), 2012}.

\item \textbf{B. Zhang, \underline{Z. Zhang$\ast$}, W. Hong, and Y. Wu, ``Applications of Distributed Compressive Sensing in Multi-channel Synthetic Aperture Radar,'' in 1st Workshop on Compressive Sensng Applied to Radar (CoSeRa 2012), 2012}.

\end{description}

~\\

\textsc{Patents}

(Bold for corresponding / communication author)

\begin{description}

\item \textbf{\underline{Z. Zhang$\ast$}, Y. Zhao, B. Zhang, W. Hong and Y. Wu, ``An Autofocus method of Sparse Microwave Imaging Radar Based on Phase Recovery,'' Chinese Patent 201310737404.4, submitted}.

\item \textbf{\underline{Z. Zhang$\ast$}, B. Zhang, W. Hong, Y. Wu and X. Quan, ``An Autofocus method of Sparse Microwave Imaging Radar Based on PhaseLift,'' Chinese Patent, submitted}.

\item B. Zhang$\ast$, W. Hong, Y. Wu and \underline{Z. Zhang}, ``Method and Device of Sparse Microwave Imaging of Imaging Radar that Installed on a Slow Motion Platform,'' Chinese Patent 201310117111.6, approved.

\item X. Quan$\ast$, B. Zhang, C. Jiang, Y. Zhao, \underline{Z. Zhang} and Y. Wu, ``A Dimension-wise Iterative Soft Threshold Sparse Microwave Imaging Method based on Sparsity Estimation,'' Chinese Patent 201410497525, submitted.

\end{description}


\end{document}